\documentclass[12pt]{article}
\usepackage{amsthm,amsmath,amsfonts,amsthm,amstext,amssymb,fullpage,framed,fancybox,graphicx,color,mdwlist,pifont, hyperref}
%\usepackage{fullpage}
\usepackage{tikz}
\def\checkmark{\tikz\fill[scale=0.4](0,.35) -- (.25,0) -- (1,.7) -- (.25,.15) -- cycle;} 
\usepackage[margin=1cm]{geometry}
\pagestyle{empty}
\newtheorem*{theorem*}{Theorem}

\newcommand{\set}[1]{\left\{ #1 \right\}}
\renewcommand{\neg}{\sim}
\begin{document}
\centerline{\bf\large Worksheet for Week 3}

\vspace{25pt}


\begin{enumerate}


\item  Prove the statement:
\begin{center}
Let $n\in\mathbb Z$. Then, $n\equiv 3 \pmod 5$ iff $5\mid (3n+1)$.
\end{center}
Hint: For one of the directions, observe that $10n-n=9n$.


\item Prove the statement:
\begin{center}
Let $a,b \in\mathbb R$. Then, if $a\neq b$, then $\dfrac{a+b}{2}>a$ or  $\dfrac{a+b}{2}>b$.
\end{center}	
Hint: How can you rewrite this statement?

\item Prove the statement:
\begin{center}
Let $x\in\mathbb R$. If $x$ is irrational, then $x^{1/3}$ is irrational.
\end{center}

% \item Prove the following statement:
% \begin{center}
% Let $x\in\mathbb R$. If $x>0$, then $x+\dfrac{2}{x}>2$.
% \end{center}

\item Prove the following statement:
\begin{center}
Let $x>0$. If $x-\dfrac{3}{x}>2$, then $x>3$.
\end{center}
Hint: Be careful with the direction of the logical implications.

Before the next examples, watch video 12 in \url{https://personal.math.ubc.ca/~PLP/auxiliary.html}. 

One important thing to note is that the usefulness of proof by cases lies in its ability to create structure we can work with, even when there is seemingly none. The next two examples are good examples of such situation.

\item Prove that if $n\in\mathbb Z$, then $n^2+3n+8$ is even.

\item Prove that if $n\in\mathbb Z$, then $2n^2+n+1$ is not divisible by $3$.

\item Let $n\in\mathbb Z$. Then prove that if $3\mid n^2$, then $3\mid n$.
Hint: You may want to use more than one proof method here.

\item Prove that for all $x\in\mathbb R$, $\left| |x+1|-|x-3| \right|\leq 4$.
Hint: How can we get rid of the absolute values in this expression?

\item Consider the riddle:
\begin{center}
We have two doors and behind one of then there is a price. There are two guardians guarding each door. One of them always tells the truth and the other always tells a lie. You are allowed to ask one question to one of the guards to figure out where the price is. What would that question be?
\end{center}

Now prove that by asking the question: ``If I ask your friend behind which door the price is, what would be their answer?'' to ANY guard, we would know where the price is.

\end{enumerate}

Before next week, watch videos 13, 14 and 15 in \url{https://personal.math.ubc.ca/~PLP/auxiliary.html}.

\end{document} 