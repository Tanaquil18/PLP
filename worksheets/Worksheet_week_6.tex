\documentclass[12pt]{article}
\usepackage{amsthm,amsmath,amsfonts,amsthm,amstext,amssymb,fullpage,framed,fancybox,graphicx,color,mdwlist,pifont, hyperref,enumitem}
%\usepackage{fullpage}
\usepackage{tikz}
\def\checkmark{\tikz\fill[scale=0.4](0,.35) -- (.25,0) -- (1,.7) -- (.25,.15) -- cycle;} 
\usepackage[margin=1cm]{geometry}
\pagestyle{empty}
\newtheorem*{theorem*}{Theorem}

\newcommand{\set}[1]{\left\{ #1 \right\}}
\renewcommand{\neg}{\sim}
\newcommand{\st}{\text{ s.t. }}
\begin{document}
\centerline{\bf\large Worksheet for Week 6}

\vspace{25pt}


\begin{enumerate}

\item Consider the sequence defined by:
$$\begin{cases} u_1=\dfrac{1}{2}&\cr
u_{n+1}= \dfrac{u_n+1}{u_n+2}&\textrm{ for $n\in\mathbb N.$ }
\end{cases}$$
Prove that $0<u_n<1$ for every $n\in \mathbb N$.

%\textbf{Proof}: We are going to use mathematical induction.
%
%\textbf{Base Case}: We see that for $n=7$, the stament is  ``$7!>3^7$". We also know that $7!=5040$ and $3^7=2187$. Thus, the statement is true for $n=7$
%
%\textbf{Inductive Step}: Let $k\geq 7$, and assume that the statement is true for $n=k$, that is, $k!>3^k$. Then we see that 
%
%\[ (k+1)!=(k+1)k!>(k+1)3^k>(3)3^k=3^{k+1}, \]
%
%since $(k+1)>3$.
%
%Therefore the statement is true for $n=k+1$, and hence, by induction, it is true for all $n\in\mathbb{N}$.
%
%


\item     The Fibonacci numbers are defined by the recurrence
$$F_1 = 1 \qquad F_2 = 1 \qquad \text{ and }\qquad F_n = F_{n - 1} + F_{n - 2}\qquad \text{ for }\qquad n > 2.$$	


Show that for every $k \in \mathbb{N}$, $F_{4k}$ is a multiple of $3$.

%\textbf{Proof}: We are going to use mathematical induction.
%
%\textbf{Base Case}: We see that for $n=1$, the stament is  ``$F_{4}$ is a multiple of $3$".  We also know that $F_3=F_2+F_1=2$, and $F_4=F_3+F_2=3$. Hence, the statement is true for $n=1$.
%
%\textbf{Inductive Step}: Let $m\geq 1$, and assume that the statement is true for $k=m$, that is, $F_{4m}$ is a multiple of $3$. Then we see that $F_{4m}=3a$ for some $a\in\mathbb{Z}$. Thus, 
%\[ F_{4(m+1)}=F_{4m+4}=F_{4m+3}+F_{4m+2}=F_{4m+2}+F_{4m+1}+F_{4m+1}+F_{4m}=F_{4m+1}+F_{4m}+2F_{4m+1}+F_{4m}=2F_{4m}+3F_{4m+1}=3(2a+F_{4m+1}), \]
%
%and since $(2a+F_{4m+1})\in\mathbb{Z}$, we see $3\mid F_{4(m+1)}$.
%
%Therefore the statement is true for $k=m+1$, and hence, by induction, it is true for all $k\in\mathbb{N}$.
%

\item Prove that $7^n-2^n$ is divisible by $5$ for all $n\in\mathbb N$.

%\begin{solution}
%We are going to prove this statement using induction.
%
%\textbf{Base case}: For $n=1$, the statement becomes, ``$7^1-2^1$ is divisible by 5". This statement is true since $5\mid 5$. Hence, the original statement is true for $n=1$.
%
%\textbf{Inductive step}: Assume that the statement is true for $a=n$ for some $n\geq 1$, that is, 
%
%$5\mid 7^n-2^n$. This means that $7^n-2^n=5m$ for some $m\in\mathbb Z$. Then,we see that 
%
%$7^{n+1}-2^{n+1}=7\cdot 7^n-2\cdot 2^n=5\cdot 7^n +2\cdot 7^n-2\cdot 2^n=5\cdot 7^n+2(7^n-2^n)=5\cdot 7^n+2\cdot 5m=5(7^n+2m)$. Hence, since $7^n+2m\in\mathbb Z$, we see that $5\mid (7^{n+1}-2^{n+1}) $.
%
%Hence, the statement is true for $a=n+1$.
%
%Therefore, by mathematical induction, we see that the statement is true for all $n\in\mathbb N$.
%
%\end{solution}

\item Let $n\in\mathbb{N}$. Prove that $\forall n\geq 7$, $n!>3^n$.


\item Let $f(x) = x \ln x$, $x > 0$ and $n\in\mathbb N$. Let $f^{(n)}(x)$ denote the $n$th derivative of $f(x)$.
Prove that if $n\geq 3$, then 

$f^{(n)}(x) = (-1)^n \dfrac{(n - 2)!}{x^{n-1}}$.

%\textbf{Proof}: We are going to use mathematical induction.
%
%\textbf{Base Case}: We see that for $n=3$, the stament is  ``$f^{(3)}(x) = - \dfrac{1}{x^{2}}$".  We see that $f'(x)=\ln(x)+1$, $f''(x)=\dfrac{1}{x}$, and thus, $f^(3)(x)=-\dfrac{1}{x^2}$. Hence, the statement is true for $n=3$.
%
%\textbf{Inductive Step}: Let $m\geq 3$, and assume that the statement is true for $n=m$, that is,  $f^{(m)}(x) = (-1)^m \dfrac{(m - 2)!}{x^{m-1}}$. Thus, 
%\[ f^{(m+1)}(x) = \dfrac{d f^{(m)}}{dx}(x) =  (-1)^{m} (1-m)\dfrac{(m - 2)!}{x^{m}}= (-1)^{m+1} (m-1)\dfrac{(m - 2)!}{x^{m}}=(-1)^{m+1}\dfrac{(m - 1)!}{x^{m}}. \]
%
%
%Therefore the statement is true for $n=m+1$, and hence, by induction, it is true for all $n\in\mathbb{N}$.
%
%

\end{enumerate}

Before the following examples, watch videos 20 and 21 in \url{https://personal.math.ubc.ca/~PLP/auxiliary.html}.


\begin{enumerate}[resume]
\item Use strong induction to prove the following statement:
Suppose you begin with a pile of $n$ stones $(n \geq 2)$ and split this pile into $n$ separate
piles of one stone each by successively splitting a pile of stones into two smaller piles.
Each time you split a pile you multiply the number of stones in each of the two smaller
piles you form, so that if these piles have $p$ and $q$ stones in them, respectively, you
compute $pq$. Show that no matter how you split the piles (eventually into $n$ piles of
one stone each), the sum of the products computed at each step equals
$\dfrac{n(n - 1)}{2}$.

For example --- say with start with $5$ stones and split them as follows:

$(5)\rightarrow \underbrace{(3)(2)}_{=6}\rightarrow \underbrace{(2)(1)}_{=2}\underbrace{(1)(1)}_{=1}\rightarrow \underbrace{(1)(1)}_{=1}(1)(1)(1)$.

Then, we get, $6+2+1+1=10=\frac{5x4}{2}\quad\checkmark$.

Hint: If we have n+1 stones, how can we separate them, and what happens then?

%\textbf{Proof}: We are going to use mathematical induction.
%
%\textbf{Base Case}: We see that for $n=2$, we have only one way of splitting them, by splitting them into two piles of one stone each. Thus, our number becomes $1\times 1=1$ and we also have $\dfrac{2(2-1)}{2}=1$. Hence, the statement is true for $n=2$.
%
%\textbf{Inductive Step}: Let $m\geq 2$, and assume that the statement is true for all $k\leq m$. Now, assume we have $m+1$ stones. Then for the first splitting, we have two cases.
%
%\textbf{Case 1: Splitting into two piles of $1$ and $m$ stones}: In this case, we have our first number to be $m\times 1=m$. Moreover, from the inductive hypothesis, we see that if we keep splitting the pile of $m$ stones we get the number $\dfrac{m(m-1)}{2}$. Thus, our final number is $m+\dfrac{m(m-1)}{2}=\dfrac{(m+1)m}{2}$. 
%
%
%Therefore, in this case, the statement is true for $n=m+1$.
%
%\textbf{Case 2: Splitting into two piles of $p$ and $q$ stones, $p,q>1$}: In this case, we have our first number to be $pq$. Moreover, since $p,q>1$, we have $p,q<m$. Thus, by inductive hypothesis, the number we should get by splitting the two piles into smaller piles are $\dfrac{p(p-1)}{2}$ and $\dfrac{q(q-1)}{2}$. We also know that $q=(m+1)-p$, and hence, our final number is 
%\[ \dfrac{p(p-1)}{2}+\dfrac{q(q-1)}{2}+pq= \dfrac{(p^2-p+q+2-q)}{2}+pq=\dfrac{p^2+q^2-(p+q)+2pq}{2},\]
%which implies 
%\[ \dfrac{(p+q)^2-(p+q)}{2}=\dfrac{(m+1)^2-(m+1)}{2}=\dfrac{(m+1)m}{2}.\]
%
%Therefore, in this case, the statement is true for $n=m+1$.
%
%Hence, we conclude that the statement is true for all $n\in\mathbb{N}$.

\item 
Let $F_k$ denote the Fibonacci sequence. Show that for every $k \geq 3$, $F_k \geq 2\cdot (3/2)^{k-3}$.


\item Show that every number $n\in \mathbb N$ can be written as $n=2^km$ where $k$ is a non-negative integer and $m$ is odd.

Hint: What happens if $n+1$ is not prime in the induction step?

\item Prove that every natural number greater than 1 has a prime factor.


\end{enumerate}

Before next week, watch videos 22 and 23 in \url{https://personal.math.ubc.ca/~PLP/auxiliary.html}.


\end{document} 