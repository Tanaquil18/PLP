\documentclass[12pt]{article}
\usepackage{amsthm,amsmath,amsfonts,amsthm,amstext,amssymb,fullpage,framed,fancybox,graphicx,color,mdwlist,pifont, hyperref}
%\usepackage{fullpage}
\usepackage{tikz}
\def\checkmark{\tikz\fill[scale=0.4](0,.35) -- (.25,0) -- (1,.7) -- (.25,.15) -- cycle;} 
\usepackage[margin=1cm]{geometry}
\pagestyle{empty}
\newtheorem*{theorem*}{Theorem}
\newcommand{\set}[1]{\left\{ #1 \right\}}
\newcommand{\s}{{\color{red} \textbf{Solution}}}

\usepackage{etoolbox}
\newtoggle{show}
%toggle this to show/hide solutions
\toggletrue{show}
%\togglefalse{show}

\newcommand{\soln}[1]{
\iftoggle{show}{
{\color{red} \textbf{Solution:}#1
}}{}
}
\begin{document}
\centerline{\bf\large Worksheet for Week 9}

\vspace{25pt}

These questions are standard, but a little challenging and long, so they should be just fine for this week.

\begin{enumerate}

\item Let $p$ be a prime number and let $a,b$ be integers. Then prove that $p\mid ab \iff (p\mid a) \text{ or } (p\mid b)$.

Hint: Bezout's lemma may be useful for one direction.

\item (old exam question) A relation $R$ on $\mathbb Z$ is defined by $aRb$ if $7a^2\equiv 2b^2 \pmod 5$. Prove that $R$ is an equivalence relation. Determine the distinct equivalence classes $[0]$ and $[1]$, simplify your answer as much as possible.\

Hint: The previous example will be useful here.



\item (old exam question)  Let $R$ be a relation on $\mathbb R$ defined as
\[ R=\set{(x,y): \cos^2(x)+\sin^2(y)=1}. \]

Prove that $R$ is an equivalence relation, and for $\theta\in\mathbb R$, write the equivalence class $[\theta]$.

Hint: For the last part of the question, you can try to visualize it on the unit circle.


\item Define a relation, S, on $\mathbb Z\times \mathbb N$ as
\[(x,y) S (a,b)\Leftrightarrow xb=ya.\]

Show that $S$  is an equivalence relation. Moreover, find the equivalence class of $(1,3)$.

Hint: How can you rewrite the relation equality?

Before the following examples, watch videos 29 and 30 in \url{https://personal.math.ubc.ca/~PLP/auxiliary.html}.

\item Suppose $P$ is a partition of a set $A$. Define a relation $R$ on $A$ by declaring $x R y$
if and only if $x, y \in X$ for some $X \in P$. Prove $R$ is an equivalence relation on $A$.
Then prove that $P$ is the set of equivalence classes of $R$.

Hint: It may be useful to draw couple disjoint sets with elements and help them visualize it that way if need be.


\item Given $n\in\mathbb{N}$, let $[a]_n$ denote the equivalence class of~$a$ under the relation ``congruence modulo~$n$" on the integers. We define the {\em multiplicative inverse} of $[a]_n$ to be the equivalence class $[b]_n$ such that $[a]_n[b]_n=[1]_n$, if such an equivalence class exists. (Multiplicative inverses are nice because they allow us to perform ``dividing by $[a]_n$'' by multiplying by the multiplicative inverse of $[a]_n$).

\begin{enumerate}
\item Write down the multiplication table for the equivalence classes of the relation ``congruence modulo~$5$". Show that every equivalence class $[k]_5$, where $k\not\equiv 0\pmod 5$, has a multiplicative inverse.

% \textbf{Solution:} $\begin{tabular}{|c|c|c|c|c|c|}
% \hline
% &\vphantom{\huge I}$\mathbf{[0]_5}$&$\mathbf{[1]_5}$&$\mathbf{[2]_5}$&$\mathbf{[3]_5}$&$\mathbf{[4]_5}$\\[10pt] \hline~&&&&&\\
% $\mathbf{[0]_5}$&$[0]_5$& $[0]_5$& $[0]_5$& $[0]_5$& $[0]_5$\\ [10pt]\hline~&&&&&\\
% $\mathbf{[1]_5}$&$[0]_5$& $[1]_5$& $[2]_5$& $[3]_5$& $[4]_5$\\ [10pt]\hline~&&&&&\\
% $\mathbf{[2]_5}$&$[0]_5$& $[2]_5$& $[4]_5$& $[1]_5$& $[3]_5$\\[10pt] \hline~&&&&&\\
% $\mathbf{[3]_5}$&$[0]_5$& $[3]_5$& $[1]_5$& $[4]_5$& $[2]_5$\\[10pt] \hline~&&&&&\\
% $\mathbf{[4]_5}$&$[0]_5$& $[4]_5$& $[3]_5$& $[2]_5$& $[1]_5$\\ [10pt] \hline
% \end{tabular}$

\vspace{10pt}
We see that multiplicative inverses of $[1]_5, [2]_5,[3]_5,[4]_5$ are $[1]_5, [3]_5,[2]_5,[4]_5$ respectively.

\item Prove that if $n\in\mathbb{N}$ is prime, then every nonzero integer modulo n has a multiplicative inverse. 

Hint: Again, Bezout's lemma.

\end{enumerate}

\item (If there is time)
Suppose that $n \in \mathbb{N}$ and $\mathbb{Z}_n$ is the set of equivalence class of congruent modulo $n$ on $\mathbb{Z}$. In this question we will call an element $[u]_n$ invertible if it has a multiplicative inverse. 

Now, define a relation $R$ on $\mathbb{Z}_n$ by $xRy$ iff  $xu=y$ for some invertible $[u]_n\in \mathbb{Z}_n$.
\begin{enumerate}
    \item Show that $R$ is a equivalence relation.
    \item Compute the equivalence classes of this relation for $n=6$.
\end{enumerate}

Hint: First find the  invertible elements in $\mathbb Z_6$.



\end{enumerate}

Before the following week, watch videos 31 and 32 in \url{https://personal.math.ubc.ca/~PLP/auxiliary.html}.

\end{document} 